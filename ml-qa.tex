\documentclass[10pt]{article}
\usepackage{amsmath, amssymb}
\usepackage[margin=0.75in, left=0.1in]{geometry}
\usepackage{longtable}
\usepackage{helvet}
\usepackage{array}
\renewcommand{\familydefault}{\sfdefault}
\renewcommand{\arraystretch}{1.4}

\setlength{\tabcolsep}{6pt}
\pagestyle{empty}

\begin{document}
\begin{center}
\section*{Bootstrapping vs. Bagging Summary}
\end{center}
\noindent
\begin{longtable}{|>{\bfseries}m{3cm}|p{6.5cm}|p{6.5cm}|}
\hline
\textbf{Feature} & \textbf{Bootstrapping} & \textbf{Bagging (Bootstrap Aggregating)} \\
\hline
\endfirsthead
\hline
\multicolumn{3}{|r|}{\textit{Table continued from previous page}} \\
\hline
\textbf{Feature} & \textbf{Bootstrapping} & \textbf{Bagging (Bootstrap Aggregating)} \\
\hline
\endhead
\hline
\multicolumn{3}{|r|}{\textit{Continued on next page}} \\
\hline
\endfoot
\hline
\endlastfoot

Primary Role & A \textbf{sampling method}. & An \textbf{ensemble learning technique}. \\
\hline

Purpose & To estimate the uncertainty of a statistic or generate data subsets. & To improve model accuracy and reduce variance (overfitting). \\
\hline

Process & Randomly sampling data \textbf{with replacement} to create a single new dataset. & \textbf{Uses bootstrapping} repeatedly to create many datasets, trains a model on each, and then aggregates the predictions. \\
\hline

Output & A new dataset (a bootstrap sample). & A single, robust final prediction. \\
\hline

\end{longtable}

\end{document}