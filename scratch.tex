\documentclass[10pt]{article}
\usepackage[utf8]{inputenc}
\usepackage{amsmath, amssymb}
\usepackage[landscape, margin=0.15in]{geometry} % Very small margin as per your previous preference
\usepackage{longtable} % For tables that can span multiple pages
\usepackage{helvet} % For sans-serif font
\usepackage{array} % For custom column types
\usepackage{enumitem} % For compact lists within table cells
\usepackage{graphicx} % Needed for \resizebox if ever used, but not for this current solution

% Redefine default font to sans-serif
\renewcommand{\familydefault}{\sfdefault}
% Adjust vertical spacing between rows in tables
\renewcommand{\arraystretch}{1.4}
% Adjust horizontal spacing between columns in tables (already quite small)
\setlength{\tabcolsep}{4pt} % Reduced from 8pt to make columns tighter

% Define a new column type for left-aligned paragraph text within table cells
\newcolumntype{L}[1]{>{\raggedright\arraybackslash}p{#1}}

\begin{document}

\section*{Statistical Test Comparison}

\noindent\footnotesize % Reduce font size for the entire table to help it fit horizontally

\begin{longtable}{|L{3.8cm}|L{3.0cm}|L{3.0cm}|L{3.0cm}|L{3.0cm}|L{3.0cm}|L{3.0cm}|L{3.0cm}|}
		% Column widths adjusted to fill page width
		% --- Header for the very first page of the table ---
		\hline
		\textbf{Feature / Test} & \textbf{Durbin-Watson (DW) Test} & \textbf{Breusch-Godfrey
		(BG) Test} & \textbf{Breusch-Pagan (BP) Test} & \textbf{White Test} &
		\textbf{Jarque-Bera (JB) Test} & \textbf{Shapiro-Wilk (SW) Test} & \textbf{Variance
		Inflation Factor (VIF)} \\
		\hline
		\endfirsthead

		% --- Header for all subsequent pages of the table ---
		\hline
		\multicolumn{8}{|r|}{\textit{Table continued from previous page}} \\
		\hline
		\textbf{Feature / Test} & \textbf{Durbin-Watson (DW) Test} & \textbf{Breusch-Godfrey
		(BG) Test} & \textbf{Breusch-Pagan (BP) Test} & \textbf{White Test} &
		\textbf{Jarque-Bera (JB) Test} & \textbf{Shapiro-Wilk (SW) Test} & \textbf{Variance
		Inflation Factor (VIF)} \\
		\hline
		\endhead

		% --- Footer for pages *before* the last page of the table ---
		\hline
		\multicolumn{8}{|r|}{\textit{Continued on next page}} \\
		\endfoot

		% --- Footer for the very last page of the table ---
		\hline
		\multicolumn{8}{|r|}{\textit{End of table}} \\
		\endlastfoot

		% --- Table Body Starts Here ---
		Purpose & Detect 1st-order autocorrelation in regression residuals. & Detect
		higher-order autocorrelation in regression residuals. & Detect heteroscedasticity
		(non-constant variance of errors). & Detect general heteroscedasticity. & Test for
		normality of residuals (based on skewness \& kurtosis). & Test for normality of
		residuals (general goodness-of-fit). & Quantify multicollinearity (how much variance of
		a coefficient is inflated). \\
		\hline
		$H_0$ (Null Hypothesis) & No 1st-order autocorrelation. & No autocorrelation up to
		specified order $p$. & Homoscedasticity (constant error variance). & Homoscedasticity
		(constant error variance). & Residuals are normally distributed. & Residuals are
		normally distributed. & (Diagnostic, no formal $H_0$) \\
		\hline
		$H_1$ (Alternative Hypothesis) & Positive or negative 1st-order autocorrelation. &
		Autocorrelation exists. & Heteroscedasticity exists. & Heteroscedasticity exists. &
		Residuals are not normally distributed. & Residuals are not normally distributed. &
		(Diagnostic, no formal $H_1$) \\
		\hline
		Test Statistic Range & 0 to 4 & Typically $\chi^2$ (Chi-squared) distributed. &
		$\chi^2$ distributed. & $\chi^2$ distributed. & $\chi^2$ distributed. & 0 to 1 & $\ge 1$ \\
		\hline
		Interpretation / Cutoff &
		\begin{itemize}[nosep, leftmargin=*, topsep=0pt, partopsep=0pt]
				\item $\approx 2$: No autocorrelation.
				\item $< 2$: Positive autocorrelation.
				\item $> 2$: Negative autocorrelation.
				\item Exact cutoffs depend on $n$ and $k$ (use tables).
		\end{itemize} &
		\begin{itemize}[nosep, leftmargin=*, topsep=0pt, partopsep=0pt]
				\item Compare p-value to $\alpha$.
				\item If p-value $<\alpha$, reject $H_0$.
		\end{itemize} &
		\begin{itemize}[nosep, leftmargin=*, topsep=0pt, partopsep=0pt]
				\item Compare p-value to $\alpha$.
				\item If p-value $<\alpha$, reject $H_0$.
		\end{itemize} &
		\begin{itemize}[nosep, leftmargin=*, topsep=0pt, partopsep=0pt]
				\item Compare p-value to $\alpha$.
				\item If p-value $<\alpha$, reject $H_0$.
		\end{itemize} &
		\begin{itemize}[nosep, leftmargin=*, topsep=0pt, partopsep=0pt]
				\item Compare p-value to $\alpha$.
				\item If p-value $<\alpha$, reject $H_0$.
		\end{itemize} &
		\begin{itemize}[nosep, leftmargin=*, topsep=0pt, partopsep=0pt]
				\item Compare p-value to $\alpha$.
				\item If p-value $<\alpha$, reject $H_0$.
				\item (Higher p-value means more evidence for normality)
		\end{itemize} &
		\begin{itemize}[nosep, leftmargin=*, topsep=0pt, partopsep=0pt]
				\item $\approx 1$: No multicollinearity.
				\item $1 < \text{VIF} < 5$: Moderate.
				\item $\ge 5$ or $\ge 10$: Significant multicollinearity.
		\end{itemize} \\
		\hline
		Common Use Cases/Context & Time series regression (checks for AR(1) errors). & Time
		series regression (more general than DW, handles AR(p) and MA(p) errors, lagged
		dependent variables). & Cross-sectional regression, time series regression. &
		Cross-sectional regression, time series regression (more robust to model
		misspecification than BP). & Checks assumption for OLS inference; often used after
		fitting. & Checks assumption for OLS inference; generally preferred for smaller
		samples. & Identifying problematic predictor variables in multiple regression. \\
		\hline
		Limitations & Only for 1st-order AR errors, doesn't work with lagged dependent
		variables. & Can be sensitive to model misspecification. & Can be sensitive to model
		misspecification. & Can be less powerful than BP if BP's assumptions hold; doesn't
		indicate source of heteroscedasticity. & Sensitive to sample size (often rejects for
		large $n$ even with minor non-normality). & Less powerful than JB for large samples;
		specific to normality. & Only detects linear relationships; doesn't fix
		multicollinearity, only identifies it. \\
		\hline
\end{longtable}

\end{document}
